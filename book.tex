%!TEX program = xelatex
\RequirePackage{xcolor}
\documentclass[
    numbers=noenddot,
    open=any,
    paper=a4,
    oneside,
    pagesize,
    captions=tableabove,
    bibliography=totoc,
    11pt
    ]{scrbook}
\usepackage[
    top=2.5cm,
    bottom=2.5cm,
    left=2.5cm,
    right=2.5cm,
    heightrounded
]{geometry}
\usepackage{amssymb,amsmath}
\usepackage[warnings-off={mathtools-colon, mathtools-overbracket}]{unicode-math}
\defaultfontfeatures{Ligatures=TeX,Scale=MatchLowercase}
\newcommand{\euro}{€}
\setmainfont[Scale=MatchLowercase,]{XITS}
\setsansfont[Scale=MatchLowercase,]{Source Sans Pro}
\setmonofont[Mapping=tex-ansi,Scale=MatchLowercase]{Hack}
\setmathfont{Asana Math}
\usepackage{upquote}
\usepackage{microtype}
\UseMicrotypeSet[protrusion]{basicmath}
% disable orphaned lines
\clubpenalty=10000
\displaywidowpenalty=10000
\widowpenalty=10000
\tolerance 1414
\hbadness 1414
\emergencystretch 1.5em
\hfuzz 0.3pt
\vfuzz 0.3pt
\raggedbottom
% more space between footnote and text
\setlength{\skip\footins}{7mm}
% footnotes
\deffootnote{2.25em}{1.75em}{\thefootnotemark\enspace}

\usepackage{polyglossia}
\setmainlanguage{english}
\setotherlanguage{ngerman}

\usepackage{csquotes}
\usepackage[
    backend=biber,
    bibstyle=biblatex-sp-unified,
    citestyle=sp-authoryear-comp,
    maxcitenames=3,
    maxbibnames=99,
    url=false,
    doi=false
]{biblatex}
\addbibresource{../bibliographyuni.bib}
\usepackage{color}
\usepackage{fancyvrb}
\newcommand{\VerbBar}{|}
\newcommand{\VERB}{\Verb[commandchars=\\\{\}]}
\DefineVerbatimEnvironment{Highlighting}{Verbatim}{commandchars=\\\{\}}
% Add ',fontsize=\small' for more characters per line
\usepackage{framed}
\definecolor{shadecolor}{RGB}{248,248,248}
\newenvironment{Shaded}{\begin{snugshade}}{\end{snugshade}}
\newcommand{\AlertTok}[1]{\textcolor[rgb]{0.94,0.16,0.16}{#1}}
\newcommand{\AnnotationTok}[1]{\textcolor[rgb]{0.56,0.35,0.01}{\textbf{\textit{#1}}}}
\newcommand{\AttributeTok}[1]{\textcolor[rgb]{0.77,0.63,0.00}{#1}}
\newcommand{\BaseNTok}[1]{\textcolor[rgb]{0.00,0.00,0.81}{#1}}
\newcommand{\BuiltInTok}[1]{#1}
\newcommand{\CharTok}[1]{\textcolor[rgb]{0.31,0.60,0.02}{#1}}
\newcommand{\CommentTok}[1]{\textcolor[rgb]{0.56,0.35,0.01}{\textit{#1}}}
\newcommand{\CommentVarTok}[1]{\textcolor[rgb]{0.56,0.35,0.01}{\textbf{\textit{#1}}}}
\newcommand{\ConstantTok}[1]{\textcolor[rgb]{0.00,0.00,0.00}{#1}}
\newcommand{\ControlFlowTok}[1]{\textcolor[rgb]{0.13,0.29,0.53}{\textbf{#1}}}
\newcommand{\DataTypeTok}[1]{\textcolor[rgb]{0.13,0.29,0.53}{#1}}
\newcommand{\DecValTok}[1]{\textcolor[rgb]{0.00,0.00,0.81}{#1}}
\newcommand{\DocumentationTok}[1]{\textcolor[rgb]{0.56,0.35,0.01}{\textbf{\textit{#1}}}}
\newcommand{\ErrorTok}[1]{\textcolor[rgb]{0.64,0.00,0.00}{\textbf{#1}}}
\newcommand{\ExtensionTok}[1]{#1}
\newcommand{\FloatTok}[1]{\textcolor[rgb]{0.00,0.00,0.81}{#1}}
\newcommand{\FunctionTok}[1]{\textcolor[rgb]{0.00,0.00,0.00}{#1}}
\newcommand{\ImportTok}[1]{#1}
\newcommand{\InformationTok}[1]{\textcolor[rgb]{0.56,0.35,0.01}{\textbf{\textit{#1}}}}
\newcommand{\KeywordTok}[1]{\textcolor[rgb]{0.13,0.29,0.53}{\textbf{#1}}}
\newcommand{\NormalTok}[1]{#1}
\newcommand{\OperatorTok}[1]{\textcolor[rgb]{0.81,0.36,0.00}{\textbf{#1}}}
\newcommand{\OtherTok}[1]{\textcolor[rgb]{0.56,0.35,0.01}{#1}}
\newcommand{\PreprocessorTok}[1]{\textcolor[rgb]{0.56,0.35,0.01}{\textit{#1}}}
\newcommand{\RegionMarkerTok}[1]{#1}
\newcommand{\SpecialCharTok}[1]{\textcolor[rgb]{0.00,0.00,0.00}{#1}}
\newcommand{\SpecialStringTok}[1]{\textcolor[rgb]{0.31,0.60,0.02}{#1}}
\newcommand{\StringTok}[1]{\textcolor[rgb]{0.31,0.60,0.02}{#1}}
\newcommand{\VariableTok}[1]{\textcolor[rgb]{0.00,0.00,0.00}{#1}}
\newcommand{\VerbatimStringTok}[1]{\textcolor[rgb]{0.31,0.60,0.02}{#1}}
\newcommand{\WarningTok}[1]{\textcolor[rgb]{0.56,0.35,0.01}{\textbf{\textit{#1}}}}

\setlength{\emergencystretch}{3em}  % prevent overfull lines
\providecommand{\tightlist}{%
    \setlength{\itemsep}{0pt}\setlength{\parskip}{0pt}}

\title{Booktitle}
\subtitle{A Fitting Subtitle (Or Not)}
\subtitle{Georg-August-University Göttingen}
\subtitle{Some Titlehead}
\subtitle{Linguistics}
\author{Maik Thalmann\footnote{Georg-August-University Göttingen,
  \href{mailto:maik.thalmann@gmail.com}{\nolinkurl{maik.thalmann@gmail.com}}}}
\date{Göttingen, 21 April, 2020}
\uppertitleback{Obiger Titelrückentitel}
\lowertitleback{Für dieses Beispiel wird keine Haftung übernommen.}

%%--------------------------

\usepackage{multicol}
\renewcommand{\arraystretch}{1.03}
\usepackage{ltablex}
\usepackage{booktabs}
\usepackage{float}

% colors
\usepackage{tikz}
\usepackage{tikz-qtree}
\usetikzlibrary{
    decorations.text, backgrounds, arrows.meta,
    decorations.pathreplacing, calc, fit,
    shapes,positioning, intersections, decorations.markings
}
\definecolor{mygreen}{RGB}{6, 107, 138} % 0, 19, 108
\definecolor{myred}{RGB}{141, 144, 150} %        96, 106, 179;
\definecolor{chaptercolor}{RGB}{188, 193, 204}
\newcommand*{\mygreen}[1]{\textcolor{mygreen}{\textbf{#1}}}
\newcommand*{\myred}[1]{\textcolor{myred}{\textbf{#1}}}

% trees in interpretation function
\newbox\mytreebox
\newbox\mytreeboxwithdelim
\makeatletter
\newcommand\TreeWithDelim [1]{%
  \setbox\mytreebox\hbox{{#1}}%
  \setbox\mytreeboxwithdelim\hbox{$\left\llbracket\vcenter{\copy\mytreebox}\right\rrbracket$}%
  \dimen@=\dimexpr\ht\mytreeboxwithdelim+\dp\mytreeboxwithdelim\relax
  % in my testing, same as total height of non decorated tree box
  \leavevmode\raise\dimexpr-.5\dimen@-\fontdimen22\textfont2+\ht\mytreebox\relax\box\mytreeboxwithdelim
}
\newcommand\TreeWithDelimParen [1]{%
  \setbox\mytreebox\hbox{{#1}}%
  \setbox\mytreeboxwithdelim\hbox{$\left(\left\llbracket\vcenter{\copy\mytreebox}\right\rrbracket\right)$}%
  \dimen@=\dimexpr\ht\mytreeboxwithdelim+\dp\mytreeboxwithdelim\relax
  % in my testing, same as total height of non decorated tree box
  \leavevmode\raise\dimexpr-.5\dimen@-\fontdimen22\textfont2+\ht\mytreebox\relax\box\mytreeboxwithdelim
}
\makeatother

\usepackage{hyperref}
\hypersetup{
    colorlinks  = true,
    urlcolor    = mygreen,
    linkcolor   = mygreen,
    citecolor   = mygreen,
    pdfauthor   = {Maik Thalmann},
    pdfproducer = {XeLaTeX},
    pdfsubject  = {Linguistics; Göttingen, 21 April, 2020},
}

% packages necessary for the semantics
\usepackage{latexsym}
\usepackage{mathtools}
\usepackage[cal=esstix]{mathalfa}
\usepackage{upgreek}
\usepackage{wasysym}
\usepackage{stmaryrd}
\usepackage{soul}
\newcommand{\sem}[1]{\mbox{$\llbracket$\sffamily\textbf{#1}$\rrbracket$}}

% checklists
\usepackage{enumitem}
\newlist{todolist}{itemize}{2}
\setlist[todolist]{label=$\square$}
\usepackage{pifont}
\newcommand{\cmark}{\ding{51}}%
\newcommand{\xmark}{\ding{55}}%
\newcommand{\done}{\rlap{$\square$}{\raisebox{2pt}{\large\hspace{1pt}\cmark}}%
    \hspace{-2.5pt}}
\newcommand{\wontfix}{\rlap{$\square$}{\large\hspace{1pt}\xmark}}

% header and footer
\usepackage{scrlayer-scrpage}
\clearscrheadfoot
% smaller header font size
\ohead{Booktitle}
\ofoot{\pagemark}
\ifoot{\url{https://mkthalmann.github.io/home/}}
\pagestyle{scrheadings}

% ToC
\BeforeTOCHead[toc]{%
  \KOMAoptions{parskip=false}
}
% change depth of TOC
\setcounter{tocdepth}{1}
% only number parts, chapters, and sections
\setcounter{secnumdepth}{1}
% custom entry formats for parts and chapters in ToC
\DeclareTOCStyleEntry[
  indent=3cm,
  beforeskip=.4cm,
  entryformat=\sffamily\large\textbf,
  pagenumberformat=\scriptsize\rmfamily\color{chaptercolor},
  linefill=\quad,
  raggedpagenumber
]{chapter}{part}
\DeclareTOCStyleEntry[
  indent=3.7cm,
  beforeskip=.1cm,
  entryformat=\sffamily,
  pagenumberformat=\scriptsize\rmfamily\color{chaptercolor},
  linefill=\quad,
  raggedpagenumber
]{chapter}{chapter}
\DeclareTOCStyleEntry[
  indent=4.2cm,
  beforeskip=.1cm,
  entryformat=\sffamily,
  pagenumberformat=\scriptsize\rmfamily\color{chaptercolor},
  linefill=\quad,
  raggedpagenumber
]{chapter}{section}

\usepackage{relsize}
\usepackage[framemethod=tikz]{mdframed}
% custom rule environment
\newmdenv[
  linecolor=chaptercolor,
  linewidth=2pt,
  topline=false,
  bottomline=false,
  rightmargin=0,
  leftmargin=0,
  innerrightmargin=5pt,
  innerleftmargin=5pt
]{rulebox}
% custom quotation environment
\newmdenv[
  linecolor=chaptercolor,
  backgroundcolor=chaptercolor!20,
  linewidth=4pt,
  font=\smaller,
  topline=false,
  bottomline=false,
  rightline=false,
  innerrightmargin=5pt,
  innerleftmargin=5pt
]{quotebox}

% part page
\newcommand*\partnumber{}
\DeclareNewLayer[
  background,
  textarea,
  mode=picture,
  contents={%
      \putC{
        \begin{tikzpicture}[remember picture,overlay,shift=(current page.north west)]
          \begin{scope}[x={(current page.north east)},y={(current page.south west)}]
            \node[align=center, anchor=center, mygreen] at (0.5, 0.35){\fontsize{30pt}{30pt}\selectfont\mdseries\partname~\thepart};
            \draw[mygreen, line width=2pt] (0.880952381, 1) -- (0.880952381, 0.9);
          \end{scope}
        \end{tikzpicture}
      }
      \gdef\partnumber{}%
    }
]{partnumber}
\DeclareNewPageStyleByLayers{part}{partnumber}
\renewcommand\partpagestyle{part}
\renewcommand*{\partformat}{\gdef\partnumber{\thepart}}

% specific font settings
\addtokomafont{part}{\fontsize{50pt}{50pt}\selectfont\textcolor{mygreen}}
\addtokomafont{pagenumber}{\color{chaptercolor}}
\renewcommand\raggedchapter{\raggedleft}
\setkomafont{chapter}{\Huge}
\setkomafont{chapterprefix}{\Large}
\newkomafont{chapternumber}{\fontsize{50pt}{10pt}\selectfont}
\addtokomafont{captionlabel}{\bfseries\color{chaptercolor}}
\addtokomafont{caption}{\selectfont\sffamily\color{chaptercolor}}
\renewcommand*{\bibfont}{\small}

\usepackage{caption}
\captionsetup[table]{position=top,skip=0pt}
\captionsetup[figure]{skip=0pt}

% chapter headings
\tikzset{
  headings/base/.style = {
      outer sep = 0pt,
      inner sep = 5pt,
    },
  headings/chapterbackground/.style = {
      headings/base,
      shade,
      left color = white,
      right color = chaptercolor,
    },
  headings/chapapp/.style = {
      headings/base,
      text = chaptercolor,
      font = \usekomafont{chapterprefix}
    },
  headings/chapternumber/.style= {
      headings/base,
      text = chaptercolor,
      font = \usekomafont{chapternumber}
    },
  headings/chapterline/.style = {
      chaptercolor,
      line width = 2pt
    }
}
\makeatletter
\renewcommand*\chapterlinesformat[3]{%
  \Ifstr{#1}{chapter}{%
    \begin{tikzpicture}[baseline=(title.base)]
      \node[headings/chapterbackground](title){%
        \parbox[t][\height]
        {\dimexpr\textwidth-3\pgfkeysvalueof{/pgf/inner xsep}\relax}
        {\raggedchapter #3}%
      };
      \node[headings/chapapp,anchor=south east, yshift=-4pt]
      at (title.north east){\Ifstr{#2}{}{}{\chapapp}};
      \useasboundingbox
      (current bounding box.north west)
      rectangle
      ([yshift=-10pt]current bounding box.south east);
      \draw[headings/chapterline]
      (current bounding box.south east)++(-.5\pgflinewidth,0)--+(0,\paperheight);
      \node[anchor=base west,headings/chapternumber]
      at([xshift=5pt]title.base-|current bounding box.east){#2};
    \end{tikzpicture}
    \par
  }{%
    \@hangfrom{#2}{#3}% other section levels using style=chapter
  }%
}
\makeatother

% section
\newkomafont{sectionnumber}{\Large}
\renewcommand\sectionformat{%
  \colorbox{chaptercolor}{%
    \enskip\usekomafont{sectionnumber}{\thesection\autodot}\enskip}%
  \quad%
}

\usepackage{gb4e}
\usepackage{etoolbox}
% roman footnotes in examples
\makeatletter
\patchcmd{\@footnotetext}{\setcounter{fnx}{0}}{\renewcommand{\thexnumi}{\roman{xnumi}}}{}{}
\apptocmd{\@footnotetext}{
    \@noftnotetrue
    \renewcommand{\thexnumi}{\arabic{xnumi}}%
}{}{}
\makeatother

\AtBeginEnvironment{Shaded}{\smaller}
\AtBeginEnvironment{verbatim}{\smaller}

% document body
\begin{document}
\maketitle

\tableofcontents
\hypertarget{a-part}{%
\part{A Part}\label{a-part}}

\hypertarget{a-chapter}{%
\chapter{A Chapter}\label{a-chapter}}

\begin{Shaded}
\begin{Highlighting}[]
\NormalTok{d \textless{}{-}}\StringTok{ }\NormalTok{diamonds}

\NormalTok{d }\OperatorTok{\%\textgreater{}\%}
\StringTok{  }\KeywordTok{head}\NormalTok{() }\OperatorTok{\%\textgreater{}\%}
\StringTok{  }\KeywordTok{kable}\NormalTok{(}\DataTypeTok{booktabs =}\NormalTok{ T, }\DataTypeTok{caption =} \StringTok{"Data we will be working with."}\NormalTok{) }\OperatorTok{\%\textgreater{}\%}
\StringTok{  }\KeywordTok{kable\_styling}\NormalTok{(}\DataTypeTok{position =} \StringTok{"center"}\NormalTok{, }\DataTypeTok{latex\_options =} \StringTok{"hold\_position"}\NormalTok{)}
\end{Highlighting}
\end{Shaded}

\begin{table}[!h]

\caption{\label{tab:data}Data we will be working with.}
\centering
\begin{tabular}[t]{rlllrrrrrr}
\toprule
carat & cut & color & clarity & depth & table & price & x & y & z\\
\midrule
0.23 & Ideal & E & SI2 & 61.5 & 55 & 326 & 3.95 & 3.98 & 2.43\\
0.21 & Premium & E & SI1 & 59.8 & 61 & 326 & 3.89 & 3.84 & 2.31\\
0.23 & Good & E & VS1 & 56.9 & 65 & 327 & 4.05 & 4.07 & 2.31\\
0.29 & Premium & I & VS2 & 62.4 & 58 & 334 & 4.20 & 4.23 & 2.63\\
0.31 & Good & J & SI2 & 63.3 & 58 & 335 & 4.34 & 4.35 & 2.75\\
\addlinespace
0.24 & Very Good & J & VVS2 & 62.8 & 57 & 336 & 3.94 & 3.96 & 2.48\\
\bottomrule
\end{tabular}
\end{table}

\hypertarget{a-section}{%
\section{A Section}\label{a-section}}

\hypertarget{another-chapter}{%
\chapter{Another Chapter}\label{another-chapter}}

In Figure \ref{tab:dlookrdiag}, we can see the amount and proportion of
missing data; and I'm checking to see how references work.

\begin{table}[!h]

\caption{\label{tab:dlookrdiag}Missing values check.}
\centering
\begin{tabular}[t]{llrrrr}
\toprule
variables & types & missing\_count & missing\_percent & unique\_count & unique\_rate\\
\midrule
carat & numeric & 0 & 0 & 273 & 0.0050612\\
cut & ordered & 0 & 0 & 5 & 0.0000927\\
color & ordered & 0 & 0 & 7 & 0.0001298\\
clarity & ordered & 0 & 0 & 8 & 0.0001483\\
depth & numeric & 0 & 0 & 184 & 0.0034112\\
\addlinespace
table & numeric & 0 & 0 & 127 & 0.0023545\\
price & integer & 0 & 0 & 11602 & 0.2150908\\
x & numeric & 0 & 0 & 554 & 0.0102707\\
y & numeric & 0 & 0 & 552 & 0.0102336\\
z & numeric & 0 & 0 & 375 & 0.0069522\\
\bottomrule
\end{tabular}
\end{table}

\begin{table}[!h]

\caption{\label{tab:dlookrdesc}Descriptive statistics for diamond price based on the diamond cut.}
\centering
\begin{tabular}[t]{llrrrrrrrr}
\toprule
variable & cut & n & na & mean & sd & se\_mean & IQR & skewness & kurtosis\\
\midrule
price & Fair & 1610 & 0 & 4358.758 & 3560.387 & 88.73281 & 3155.25 & 1.783535 & 3.088025\\
price & Good & 4906 & 0 & 3928.864 & 3681.590 & 52.56197 & 3883.00 & 1.722996 & 3.049343\\
price & Very Good & 12082 & 0 & 3981.760 & 3935.862 & 35.80721 & 4460.75 & 1.595738 & 2.238162\\
price & Premium & 13791 & 0 & 4584.258 & 4349.205 & 37.03497 & 5250.00 & 1.333648 & 1.073710\\
price & Ideal & 21551 & 0 & 3457.542 & 3808.401 & 25.94233 & 3800.50 & 1.835843 & 2.978950\\
\bottomrule
\end{tabular}
\end{table}

\begin{table}[!h]

\caption{\label{tab:report}Data overview.}
\centering
\resizebox{\linewidth}{!}{
\begin{tabular}[t]{lllllll}
\toprule
Characteristic & Fair (n=1610) & Good (n=4906) & Very Good (n=12082) & Premium (n=13791) & Ideal (n=21551) & Total\\
\midrule
Mean price (SD) & 4358.8 (3560.4) & 3928.9 (3681.6) & 3981.8 (3935.9) & 4584.3 (4349.2) & 3457.5 (3808.4) & 3932.8 (3989.4)\\
\bottomrule
\end{tabular}}
\end{table}

\hypertarget{another-section}{%
\section{Another Section}\label{another-section}}

\hypertarget{yet-another-section}{%
\section{Yet Another Section}\label{yet-another-section}}

\begin{figure}

{\centering \includegraphics[width=0.8\linewidth]{figs/mean-lines-1} 

}

\caption{A sample plot.}\label{fig:mean-lines}
\end{figure}

\hypertarget{another-part}{%
\part{Another Part}\label{another-part}}

\hypertarget{some-additional-chapter}{%
\chapter{Some Additional Chapter}\label{some-additional-chapter}}

\textcite{tarski1944semantic}

\textcite{dayal2009variation}

\hypertarget{all-the-code-i-used}{%
\chapter*{All The Code I Used}\label{all-the-code-i-used}}

\begin{Shaded}
\begin{Highlighting}[]
\NormalTok{d \textless{}{-}}\StringTok{ }\NormalTok{diamonds}

\NormalTok{d }\OperatorTok{\%\textgreater{}\%}
\StringTok{  }\KeywordTok{head}\NormalTok{() }\OperatorTok{\%\textgreater{}\%}
\StringTok{  }\KeywordTok{kable}\NormalTok{(}\DataTypeTok{booktabs =}\NormalTok{ T, }\DataTypeTok{caption =} \StringTok{"Data we will be working with."}\NormalTok{) }\OperatorTok{\%\textgreater{}\%}
\StringTok{  }\KeywordTok{kable\_styling}\NormalTok{(}\DataTypeTok{position =} \StringTok{"center"}\NormalTok{, }\DataTypeTok{latex\_options =} \StringTok{"hold\_position"}\NormalTok{)}
\NormalTok{d }\OperatorTok{\%\textgreater{}\%}
\StringTok{  }\KeywordTok{diagnose}\NormalTok{() }\OperatorTok{\%\textgreater{}\%}
\StringTok{  }\KeywordTok{kable}\NormalTok{(}\DataTypeTok{booktabs =}\NormalTok{ T, }\DataTypeTok{caption =} \StringTok{"Missing values check."}\NormalTok{) }\OperatorTok{\%\textgreater{}\%}
\StringTok{  }\KeywordTok{kable\_styling}\NormalTok{(}\DataTypeTok{position =} \StringTok{"center"}\NormalTok{, }\DataTypeTok{latex\_options =} \StringTok{"hold\_position"}\NormalTok{)}
\NormalTok{d }\OperatorTok{\%\textgreater{}\%}
\StringTok{  }\KeywordTok{group\_by}\NormalTok{(cut) }\OperatorTok{\%\textgreater{}\%}
\StringTok{  }\KeywordTok{describe}\NormalTok{(price) }\OperatorTok{\%\textgreater{}\%}
\StringTok{  }\KeywordTok{select}\NormalTok{(}\OperatorTok{{-}}\KeywordTok{starts\_with}\NormalTok{(}\StringTok{"p"}\NormalTok{)) }\OperatorTok{\%\textgreater{}\%}
\StringTok{  }\KeywordTok{kable}\NormalTok{(}\DataTypeTok{booktabs =}\NormalTok{ T, }\DataTypeTok{caption =} \StringTok{"Descriptive statistics for diamond price based on the diamond cut."}\NormalTok{) }\OperatorTok{\%\textgreater{}\%}
\StringTok{  }\KeywordTok{kable\_styling}\NormalTok{(}\DataTypeTok{position =} \StringTok{"center"}\NormalTok{, }\DataTypeTok{latex\_options =} \StringTok{"hold\_position"}\NormalTok{)}
\NormalTok{d }\OperatorTok{\%\textgreater{}\%}
\StringTok{  }\KeywordTok{report\_sample}\NormalTok{(}\DataTypeTok{group\_by =} \StringTok{"cut"}\NormalTok{, }\DataTypeTok{select =} \KeywordTok{c}\NormalTok{(}\StringTok{"price"}\NormalTok{)) }\OperatorTok{\%\textgreater{}\%}
\StringTok{  }\KeywordTok{as.data.frame}\NormalTok{() }\OperatorTok{\%\textgreater{}\%}
\StringTok{  }\KeywordTok{kable}\NormalTok{(}\DataTypeTok{booktabs =}\NormalTok{ T, }\DataTypeTok{caption =} \StringTok{"Data overview."}\NormalTok{) }\OperatorTok{\%\textgreater{}\%}
\StringTok{  }\KeywordTok{kable\_styling}\NormalTok{(}\DataTypeTok{position =} \StringTok{"center"}\NormalTok{, }\DataTypeTok{latex\_options =} \KeywordTok{c}\NormalTok{(}\StringTok{"scale\_down"}\NormalTok{, }\StringTok{"hold\_position"}\NormalTok{))}
\KeywordTok{ggplot}\NormalTok{(d, }\KeywordTok{aes}\NormalTok{(}\DataTypeTok{x =}\NormalTok{ cut, }\DataTypeTok{y =}\NormalTok{ price, }\DataTypeTok{color =}\NormalTok{ price)) }\OperatorTok{+}
\StringTok{  }\KeywordTok{stat\_summary}\NormalTok{(}\DataTypeTok{fun.y =}\NormalTok{ mean, }\DataTypeTok{geom =} \StringTok{"line"}\NormalTok{, }\DataTypeTok{mapping =} \KeywordTok{aes}\NormalTok{(}\DataTypeTok{group =} \DecValTok{1}\NormalTok{)) }\OperatorTok{+}
\StringTok{  }\KeywordTok{stat\_summary}\NormalTok{(}\DataTypeTok{fun.y =}\NormalTok{ mean, }\DataTypeTok{geom =} \StringTok{"point"}\NormalTok{) }\OperatorTok{+}
\StringTok{  }\KeywordTok{stat\_summary}\NormalTok{(}\DataTypeTok{fun.data =}\NormalTok{ mean\_cl\_normal, }\DataTypeTok{geom =} \StringTok{"errorbar"}\NormalTok{, }\DataTypeTok{width =} \FloatTok{.25}\NormalTok{) }\OperatorTok{+}
\StringTok{  }\KeywordTok{theme\_maik}\NormalTok{() }\OperatorTok{+}
\StringTok{  }\KeywordTok{scale\_color\_grey}\NormalTok{()}
\end{Highlighting}
\end{Shaded}


    \printbibliography[heading=bibintoc]


\end{document}